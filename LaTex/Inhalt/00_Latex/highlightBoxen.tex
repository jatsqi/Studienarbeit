% ---- Definition für Highlight Boxen

% ---- Grundsätzliche Definition zum Style
\newtheoremstyle{defi}
  {\topsep}         % Abstand oben
  {\topsep}         % Abstand unten
  {\normalfont}     % Schrift des Bodys
  {0pt}             % Einschub der ersten Zeile
  {\bfseries}       % Darstellung von der Schrift in der Überschrift
  {:}               % Trennzeichen zwischen Überschrift und Body
  {.5em}            % Abstand nach dem Trennzeichen zum Body Text
  {\thmname{#3}}    % Name in eckigen Klammern
\theoremstyle{defi}

% ------ Definition zum Strich vor eines Texts
\newmdtheoremenv[
  hidealllines = true,       % Rahmen komplett ausblenden
  leftline = true,           % Linie links einschalten
  innertopmargin = 0pt,      % Abstand oben
  innerbottommargin = 4pt,   % Abstand unten
  innerrightmargin = 0pt,    % Abstand rechts
  linewidth = 3pt,           % Linienbreite
  linecolor = gray!40,       % Linienfarbe
]{defStrich}{Definition}     % Name der des formats "defStrich"

% ------ Definition zum Eck-Kasten um einen Text
\newmdtheoremenv[
  hidealllines = true,
  innertopmargin = 6pt,
  linecolor = gray!40,
  singleextra={              % Eck-Markierungen für die Definition
    \draw[line width=3pt,gray!50,line cap=rect] (O|-P) -- +(1cm,0pt);
    \draw[line width=3pt,gray!50,line cap=rect] (O|-P) -- +(0pt,-1cm);
    \draw[line width=3pt,gray!50,line cap=rect] (O-|P) -- +(-1cm,0pt);
    \draw[line width=3pt,gray!50,line cap=rect] (O-|P) -- +(0pt,1cm);
  }
]{defEckKasten}{Definition}  % Name der des formats "defEckKasten"